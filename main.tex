\documentclass[a4paper, 12pt]{article}

\usepackage[utf8]{inputenc}
\usepackage[T1]{fontenc}
\usepackage[ngerman]{babel}
\usepackage{amsmath}
\usepackage{mathtools}
\begin{document}
\title{Mögliche Alternativen zur Allgemeinen Relativitäthstheorie}
\author{Maximilian Schubert}
\date{}
\maketitle
\section{Einleitung} 

Eine der bedeutensten wissenschaftlichen Vorlesungen im Bereich der Physik (des 20ten Jahrhundert) trug der theoretische Physiker und spätere Nobelpreisträger Albert Einstein am 25. November 1915 an der Preußischen Akademie der Wisschenschaften vor. Durch (die Idee einer gekrümmten Raumzeit und derer Formulierung) eine geometrische Formulierung der gekrümmten Raumzeit konnte (er) die Gravitation über das bis dahin gängige newtonische Weltbild hinaus beschreiben (beschrieben werden). Der am 20. März 1916 in den " Annalen der Physik" veröffentliche Artikel "Die Grundlage der allgemeinen Relativitätstheorie"  bietet die erste Grundlage für die einsteinische (korrekte?) Interpretation der Raumzeit. 

Die vorliegende Arbeit wird sich in Kapitel 2 mit der Frage auseinandersetzten, welche Aspekte der ART in allternativ Theorien übernommen werden müssen und welche angepasst werden können. Im Folgenen 
\section{Einheitliche Feldtheorien}
 // Definition einer Feldtheorie\\
 // Bedeutung einer einheitlichen Feldtheorie\\
 
\subsection{Stringtheorie}
// Grundlagen\\

\subsubsection{M-Theorie}

\subsection{Schleifenquantengravitation}

\subsection{Supergravitation}

\section{Die Allgemeine Relativitätstheorie in der \\ Stringtheorie}

\subsection{Lichtkegel Koordinaten/Light-cone coordinates}

In special relativity, events are characterized by the values of four coordinates: a time coordinate \textit{t} and three spatial coordinates \textit{x, y} and \textit{z}, [...] where time is scaled by the speed of light \textit{c} so that all coordinates have units of length. $$ x^ \mu = (x^0,x^1,x^2,x^3) \equiv (ct,x,y,z)$$
The $x^ \mu$ are spacetime coordinates (vgl. Zwiebach 2009, 15)

\subsection{Relativistische Strings}

\subsubsection{Offene relativistische Strings}

\subsubsection{Geschlossene relativistische Strings}

\subsection{World-sheets}

\section{Nicht-relativistische Aspekte der \\ Stringtheorie}

\subsection{Nicht-Relativistische Strings}

\section{Verifizierungsprobleme der Stringtheorie}

\section{Fazit}

\begin{thebibliography}{9}

\bibitem{einstein}
Einstein, Albert. 1905. 
\textit{Zur Elektrodynamik bewegter Körper.}
891-921

\bibitem{zwiebach}
Zwiebach, Barton. 2009.
\textit{A First Course in Stringtheory.}
Haryana: Rajkamal Electric Press.
\bibitem{gubser}
Gubser, Steven. 2011. 
\textit{Das Kleine Buch der Stringtheorie.}
Heidelberg: Spektrum Akademischer Verlag
\bibitem{zeilinger}
Zeilinger, Anton. 2003.
\textit{Einsteins Schleier: die neue Welt der Quantenphysik.}
München: Wilhelm Goldmann
\end{thebibliography}

\end{document}