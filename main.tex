\documentclass[a4paper, 12pt]{article}

\usepackage[utf8]{inputenc}
\usepackage[T1]{fontenc}
\usepackage[ngerman]{babel}
\usepackage{amsmath}
\usepackage{mathtools}
\begin{document}
\title{Mögliche Alternativen zur Allgemeinen Relativitäthstheorie}
\author{Maximilian Schubert}
\date{}
\maketitle
\section{Einleitung} 

Eine der bedeutensten wissenschaftlichen Vorlesungen im Bereich der Physik (des 20ten Jahrhundert) trug der theoretische Physiker und spätere Nobelpreisträger Albert Einstein am 25. November 1915 an der Preußischen Akademie der Wisschenschaften vor. Durch (die Idee einer gekrümmten Raumzeit und derer Formulierung) eine geometrische Formulierung der gekrümmten Raumzeit konnte (er) die Gravitation über das bis dahin gängige newtonische Weltbild hinaus beschreiben (beschrieben werden). Der am 20. März 1916 in den " Annalen der Physik" veröffentliche Artikel "Die Grundlage der allgemeinen Relativitätstheorie"  bietet die erste Grundlage für die einsteinische (korrekte?) Interpretation der Raumzeit. 

Die vorliegende Arbeit wird sich in Kapitel 2 mit der Frage auseinandersetzten, welche Aspekte der ART in allternativ Theorien übernommen werden müssen und welche angepasst werden können. Im Folgenen 
\section{Renormierungsproblem}
\section{Einheitliche Feldtheorien}
 // Definition einer Feldtheorie\\
 // Bedeutung einer einheitlichen Feldtheorie\\
 Nach der Veröffentlichung und erfolgreichen experimentellen Verifizierung der Allgemeinen Relativitätstheorie suchte Einstein bis zu seinem Lebensende nach einer "Theorie von Allem" (englisch "Theory of Everything"). Dabei versuchte er das Problem einer solchen einheitlichen Feldtheorie mithilfe des hamiltonschen Prinzips zu lösen und veröffentlicht seine Ideen in der am 21. März erschienenen Arbeit "Einheitliche Feldtheorie und Hamiltonsches Prinzip" (vgl. Einstein 1929, 156). Einstein ging der Kaluza-Klein-Theorie nach, die als \textit{one of the first attempts to create a unified field theory} gilt. \textit{It was publishied in 1921 by German mathematician and physicist Theodor Kaluza and extended in 1926 by Oskar Klein.}(vgl. Zalo\v{z}nik 2012, 2)
 
\subsection{Stringtheorie}
Grundlagen
\begin{quote}
When we speak of parameters in a theory, it is convenient to distinguish between dimensionful parameters and dimensionless parameters. [...] Stringtheory is said to have no adjustable parameters, meaning that no dimensionless parameter is needed to formulate string theory. String theory does, however, have one dimensionful parameter. That parameter is the string length $\ell_s$. (vgl. Zwiebach 2009, 14)\\
\end{quote}

\noindent "Dimensonless" bzw. "dimensonful parameters" 
\subsubsection{M-Theorie}

\subsection{Schleifenquantengravitation}

\subsection{Supergravitation}

\section{Die Allgemeine Relativitätstheorie in der \\ Stringtheorie}

\subsection{Lichtkegel Koordinaten/Light-cone coordinates}
\begin{quote}

In special relativity, events are characterized by the values of four coordinates: a time coordinate \textit{t} and three spatial coordinates \textit{x, y} and \textit{z}, [...] where time is scaled by the speed of light \textit{c} so that all coordinates have units of length. $$ x^ \mu = (x^0,x^1,x^2,x^3) \equiv (ct,x,y,z)$$
The $x^ \mu$ are spacetime coordinates (vgl. Zwiebach 2009, 15)
\end{quote}
\subsection{Relativistische Strings}

Dieses Kapitel beschäfg

\subsubsection{Offene relativistische Strings}

\subsubsection{Geschlossene relativistische Strings}

\subsection{World-sheets}

\section{Nicht-relativistische Aspekte der \\ Stringtheorie}

\subsection{Nicht-Relativistische Strings}

\section{Verifizierungsprobleme der Stringtheorie}

\section{Fazit}

\begin{thebibliography}{9}

\bibitem{einstein}
Einstein, Albert. 1905. 
\textit{Zur Elektrodynamik bewegter Körper.}
891-921

\bibitem{zwiebach}
Zwiebach, Barton. 2009.
\textit{A First Course in Stringtheory.}
Haryana: Rajkamal Electric Press.
\bibitem{gubser}
Gubser, Steven. 2011. 
\textit{Das Kleine Buch der Stringtheorie.}
Heidelberg: Spektrum Akademischer Verlag
\bibitem{zeilinger}
Zeilinger, Anton. 2003.
\textit{Einsteins Schleier: die neue Welt der Quantenphysik.}
München: Wilhelm Goldmann
\bibitem{zaloznik}
Zalo\v{z}nik, An\v{z}e. 2012." Kaluza-Klein Theory". Aufgerufen am 11. Dezember 2018.\\
http://mafija.fmf.uni-lj.si/seminar/files/2011\_2012/KaluzaKlein\_theory.pdf
\end{thebibliography}

\end{document}